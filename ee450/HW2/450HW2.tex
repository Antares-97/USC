\documentclass[31pt]{article}
\setlength{\columnsep}{0.13\columnwidth}
%\setlength{\columnseprule}{0.05\columnwidth}
 
\usepackage{geometry}
\geometry{left=2.3cm, right=2.3cm, top=2.5cm, bottom=2.5cm}

\usepackage{fontspec}
\setmainfont{Times New Roman}

\usepackage{xeCJK}

\usepackage{subfigure}

\usepackage{amssymb}

\usepackage{amsmath}

\usepackage{graphicx}

\usepackage{booktabs}

\usepackage{longtable}

\usepackage{tabularx}

\usepackage{wrapfig}

\usepackage{indentfirst}

\usepackage{bm}								%粗斜体

\usepackage{float}								%超级好用!浮动排版!

\usepackage{flushend,cuted}
 
\usepackage{caption}
\captionsetup{font={scriptsize}}						%改变图名字体大小

\usepackage{subfig}
\captionsetup[subfigure]{labelformat=simple, listofformat=subsimple, farskip = 0pt}

\usepackage{hyperref}							%超链接!

\usepackage{fancyhdr}

\usepackage{stfloats}

\setlength{\parindent}{2em}

\linespread{1.2}

\usepackage{ragged2e}             %两端对齐!

\usepackage{algorithm}

\usepackage{algorithmicx}

\usepackage{algpseudocode}

\renewcommand{\algorithmicrequire}{\textbf{Input:}}  % Use Input in the format of Algorithm
\renewcommand{\algorithmicensure}{\textbf{Output:}} % Use Output in the format of Algorithm

\begin{document}

\title{EE450 Introduction to Computer Networks - Fall 2019 - HW 2}
\author{Junzhe Liu,\; 2270250947}

\begin{document}

\maketitle

\pagestyle{fancy}
\lhead{}
\rhead{\textbf{\thepage}}
\chead{\textit{ Junzhe Liu / 2270250947 / Viterbi School of Engineering, Computer Science}}
\lfoot{}
\cfoot{}
\rfoot{}

\section{Reading Assignment:}

\justifying\large Chapter 1

\section{Problems to be solved:}

\subsection{Chapter 1, Page 69: R19 Suppose Host A wants to send a large file to Host B. The path from Host A to Host B has three links, of rates $R_1 = 500$ Kbps, $R_2 = 2$ Mbps, and $R_3 = 1$ Mbps.}

\subsubsection{Assuming no other traffic in the network, what is the throughput for the file transfer?}

\begin{equation}
Throughput = \min(R_1, R_2, R_3)= 500\; \mathrm{Kbps}.
\end{equation}

\subsubsection{Suppose the file is 4 million bytes. Dividing the file size by the throughput, roughly how long will it take to transfer the file to Host B?}

\begin{equation}
Time = \frac{4 \;\mathrm{million\;bytes}\times 8\mathrm{bits}}{500 \;\mathrm{Kbps}}= 64\; s.
\end{equation}

\subsubsection{Repeat (a) and (b), but now with R2 reduced to 100 kbps.}

\begin{align}
&Throughput^{'} = 100\; \mathrm{Kbps}.\\
&Time^{'}=\frac{4 \;\mathrm{million\;bytes}\times 8\mathrm{bits}}{100 \;\mathrm{Kbps}} = 320\;s.
\end{align}

\subsection{Chapter 1, Page 69: R23 What are the five layers in the Internet protocol stack? What are the principal responsibilities of each of these layers?}

Top-down:
\begin{itemize}
\item Application Layer: supporting network applications

\item Transport Layer: process-process data transfer

\item Network Layer: routing of datagrams from source to destination

\item Link Layer: data transfer between neighboring  network elements

\item Physical Layer: bits "on the wire"
\end{itemize}


\subsection{Chapter 1, Page 71: P5 Review the car-caravan analogy in Section 1.4. Assume a propagation speed of 100 km/hour.}

\subsubsection{Suppose the caravan travels 150 km, beginning in from one tollbooth, passing through a second tollbooth, and finishing just after a third tollbooth. What is the end-to-end delay?}

The caravan is comprised of 10 cars. Each tollbooth are 75 km apart, cars' speed are 100 km/h, tollbooths service cars at a rate of one car per every 12 seconds.

Tollbooth operation time: $10\times 0.2$=2\;min.

Propagation time: $75/100$=0.75\;h=45\; min.

Total time: $2\times3+45\times 2$=96 \;min.


\subsubsection{Repeat (a) now assuming that there are eight cars in the caravan instead of ten.}

Tollbooth operation time is now: $8\times 0.2$=1.6\; min.

Total time: 94.8 \;min


\subsection{Chapter 1, Page 72: P10 }

Total end-to-end delay: 
\begin{equation}
\frac{d_1}{s_1}+\frac{L}{R_1}+d_{proc}+\frac{d_2}{s_2}+\frac{L}{R_2}+d_{proc}+\frac{d_3}{s_3}+\frac{L}{R_3} = 2d_{proc}+\frac{d_1}{s_1}+\frac{d_2}{s_2}+\frac{d_3}{s_3}+L\left(\frac{1}{R_1}+\frac{1}{R_2}+\frac{1}{R_3}\right)
\end{equation}

applying real values: Total delay = 64 msec.

\subsection{Chapter 1, Page 75: P24  Suppose you would like to urgently deliver 40 terabytes data from Boston to Los Angeles. You have available a 100 Mbps dedicated link for data transfer. Would you prefer to transmit the data via this link or instead use FedEx overnight delivery? Explain.}

40 Terabytes = $40\times 10^{12}\times 8=3.2\times10^{14}$ bits. Total time will be $3.2\times10^{14}\,bits/\,100\;Mbps\,=\,3.2\times10^{6}\;$s\;=\;$8.9\times 10^{2}$\;h\,=37 days.

But with FedEx, data can be guaranteed to be delivered in one day. So I choose FedEx.




\end{document}